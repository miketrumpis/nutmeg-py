%% LyX 1.6.4.1 created this file.  For more info, see http://www.lyx.org/.
%% Do not edit unless you really know what you are doing.
\documentclass[english]{article}
\usepackage[T1]{fontenc}
\usepackage[latin9]{inputenc}
\usepackage{babel}

\begin{document}

\section{Nutmeg In Python}

Python is an open-source, general purpose, object oriented programming
language that is gaining popularity as a tool for scientific computing.
As an interpreted language with robust object model support, Python
allows a wide variety of programming styles, from line-by-line scripting
to abstracted, reusable library code. Though not specifically developed
for numerics and scientific computation, its strenghts include an
emphasis on legibility and ease-of-use, system portability, and straightforward
access to system libraries. Additionally, there is a very stable stack
of basic computational tools actively developed by the scientific
Python community. First among the many commonly used tools are: Numpy
for n-dimensional arrays; Scipy for a wealth of computational code,
much of it being a Python layer over established, validated libraries
such as LAPACK and FFTPACK; and Matplotlib, which provides interactive
and scriptable 2D plotting tools that emulate MATLAB plotting. All
these features provide a convenient computing environment for the
development of modern scientific data processing systems, whose scope
may expand over time, and whose core functionality typically demand
a design covering a range from optimized algorithms to complex <\textcompwordmark{}<
complicated? >\textcompwordmark{}> data models for physical phenomena.

Nutmeg has begun a small scale transition from MATLAB to Python, named
Nutmeg-Py. To date, the membrane from one system to the other lies
between voxelwise source reconstruction, and statistical post processing
and visualization. 


\subsection{From MATLAB Data To Python Objects}

The workflow for a Nutmeg based analysis that incorporates Python
tools presents both a design challenge and a technical data translation
problem. The latter is a solved problem, thanks to code from SciPy
enabling I/O between Numpy arrays and MATLAB data contained in {}``mat''
files. The former allows the use of Python's object model. 

Nutmeg-Py's core includes very simple data models which, abstractly,
have immutable data and metadata, have methods to interrogate or transform
the data in some fashion, and finally can read and write itself on
disk without loss of precision. The TFBeam is an example of such an
object, and is the Python analog to the MATLAB {}``struct'' containing
a time-frequency reconstruction. Modeling the data in this fashion
has the advantage of allowing users to more easily develop custom
scripts and routines to further interrogate their results.

The toolbox side of Nutmeg-Py currently includes a non-parametric
statistical testing package, based on <\textcompwordmark{}< Nichols
and Holmes, 2001 >\textcompwordmark{}>, including cluster level analysis
from <\textcompwordmark{}< Hayasaka and Nichols, 2004 >\textcompwordmark{}>.
Both approaches have been adapted to the five-dimensional space of
time-frequency MEG imaging. The results are encapsulated in an object
oriented manner, as the TimeFreqSnPMResults, which stores the generated
null distributions, and has methods available for creating thresholds
and maps based on levels of significance.

Another common design pattern employed in Nutmeg-Py is the modeling
of a related set of processing task using a hierarchy of classes.
This is done by defining a common parent classes that specifies or
even performs the bulk of the data organization and computation, and
then defining variations of the task with subclasses that prepare
data in task-specific ways and inherit the common functionality. Nutmeg-Py
takes advantage of this pattern in its recreation of Nutmeg's non-parametric
statistical processing machinery. 


\subsection{Visualization}

The project to transition Nutmeg into a Python based toolkit has also
generated a small but powerful visualization effort named Xipy (cross-modality
imaging in Python), which lies somewhat under the umbrella of the
seminal Nipy (neuroimaging in Python) project. The main ambition of
Xipy is to provide a flexible and extensible system for displaying
brain imagery from various data sources (eg, anatomical MR, statistical
maps, diffusion tracks) in the same 3D scene (see figure X). The graphics
technology behind most displays in Xipy is Mayavi <\textcompwordmark{}<
ref? >\textcompwordmark{}>, which is a Python visualization code base
which itself uses VTK <\textcompwordmark{}< ref? >\textcompwordmark{}>
for all of its graphical heavy lifting.

Xipy is in fact totally decoupled from Nutmeg-Py. Visualization of
results from Nutmeg and Nutmeg-Py within Xipy is enabled by a richly
featured plugin contained in the Nutmeg-Py package. The plugin itself
is a custom implementation of comparatively simple, stereotyped overlay
and thresholding interfaces. In principle, this overlay plugin architecture
could be exploited for displaying any dataset that can be mapped to
a volumetric grid coregistered to an underlying image. 
\begin{thebibliography}{2}
\bibitem{key-1}Satoru Hayasaka and Thomas E. Nichols, Combining voxel
intensity and cluster extent with permutation test framework, NeuroImage
23 (2004) 54\textendash{}63

\bibitem{key-2}Thomas E. Nichols and Andrew P. Holmes, Nonparametric
Permutation Tests For Functional Neuroimaging: A Primer with Examples,
Human Brain Mapping 15 1\textendash{}25 (2001)
\end{thebibliography}

\end{document}
